\subsection{Fragestellung}

Die vorliegende Arbeit untersucht die Implementierung des Equilibrium Propagation Algorithmus auf analogen Computern und die damit verbundenen Herausforderungen und Potenziale. Dabei stellt sich die Forschungsfrage:

Wie kann der Equilibrium Propagation Algorithmus effektiv auf analogen Computern implementiert werden und welche spezifischen Herausforderungen ergeben sich aus dieser Umsetzung?

Der Fokus liegt auf der praktischen Umsetzung des Algorithmus in einer Simulationsumgebung und der notwendigen Anpassung des Hopfield-Netzerks für die Anwendung von Equilibrium-Propagation. Dabei werden insbesondere die Herausforderungen bei der Übernahme theoretischer Modelle in analoge Rechenstrukturen betrachtet. Zudem wird analysiert, inwiefern analoge Systeme durch ihre kontinuierliche Signalverarbeitung eine effiziente Alternative zu digitalen Implementierungen bieten können.
