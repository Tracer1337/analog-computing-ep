\subsection{Zielsetzung der Arbeit}

Ziel dieser Arbeit ist die Implementierung des \ac{ep} Algorithmus auf einem analogen Computer und die Untersuchung der damit verbundenen Herausforderungen und Potenziale. Dabei wird analysiert, inwiefern sich der Algorithmus an analoge Rechenstrukturen anpassen lässt und welche Vorteile sich daraus für das Training neuronaler Netze ergeben.

Ein zentraler Aspekt ist die Modifikation eines \ac{hnn} zur Nutzung von \ac{ep}. Dazu wird das Netzwerk in einer Simulationsumgebung abgebildet und entsprechend angepasst. Im Fokus stehen dabei die mathematischen und technischen Anforderungen für eine stabile Implementierung sowie mögliche Einschränkungen durch die Eigenschaften analoger Systeme.

Besondere Herausforderungen ergeben sich aus der kontinuierlichen Signalverarbeitung analoger Rechner, wodurch klassische Optimierungsmethoden, wie sie in digitalen Systemen genutzt werden, nicht direkt übertragbar sind. Daher wird untersucht, welche strukturellen Anpassungen erforderlich sind und wie sich analoge Komponenten für die Implementierung des Lernverfahrens nutzen lassen.

Die Arbeit soll damit einen Beitrag zur Evaluierung analoger Rechenarchitekturen im Kontext maschinellen Lernens leisten und aufzeigen, inwiefern \ac{ep} auf nicht-digitalen Systemen realisierbar ist.
