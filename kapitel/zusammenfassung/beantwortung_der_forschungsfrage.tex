\subsection{Beantwortung der Forschungsfrage}

Die Forschungsfrage dieser Arbeit lautet: Wie kann der \gls{eqprop} Algorithmus effektiv auf analogen Computern implementiert werden und welche spezifischen Herausforderungen ergeben sich aus dieser Umsetzung?

Die durchgeführte Konstruktion zeigt, dass das \gls{c-ep} als Abwandlung des \gls{eqprop} grundsätzlich als analoger Rechner realisierbar ist. In der Simulationsumgebung kann der zweiphasige Lernprozess erfolgreich nachgebildet werden, wobei stabile Fixpunkte im \gls{hopfieldnetzwerk} erreicht werden. Die kontinuierliche Anpassung der Gewichte erweist sich als umsetzbar, obwohl sie eine diskrete Annäherung erfordert, da eine mathematische Beschreibung der Differentialgleichung für die Gewichtsanpassung bislang nicht vorliegt.

Herausforderungen ergeben sich insbesondere bei der Wahl der Aktivierungsfunktion. Während die Sigmoid-Funktion in der Simulation verwendet wurde, bleibt offen, ob sie sich für eine physikalische Implementierung optimal eignet. Zudem beeinflusst die Wahl des Einflussparameters \(\beta\) und der Lernrate \(\eta\) maßgeblich die Stabilität und Geschwindigkeit der Konvergenz, wobei zu hohe Werte Instabilitäten verursacht und zu niedrige die Effizienz reduziert.

Die Skalierbarkeit des Ansatzes auf tiefere Netzwerke bleibt eine offene Fragestellung. Während die Implementierung für ein einfaches \gls{hopfieldnetzwerk} funktioniert, sind weitere Untersuchungen erforderlich, um die Tragfähigkeit des \gls{c-ep} für größere Netzwerke zu bewerten. Analoge Hardware könnte dabei sowohl Einschränkungen als auch Vorteile bieten, die noch nicht abschließend erforscht wurden.
