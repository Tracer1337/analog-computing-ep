\subsection{Herausforderungen und Lösungsansätze}

Die Implementierung des \ac{c-ep} Algorithmus auf analogen Computern bringt eine Reihe technischer und konzeptioneller Herausforderungen mit sich. Ein Problem stellt die Wahl einer geeigneten Aktivierungsfunktion dar. Die ursprünglich eingesetzte Sigmoid-Funktion nähert sich den Grenzwerten \(0\) bzw. \(1\), was zu einer fehlerhaften Darstellung dieser Werte im Netzwerk führt. Als Alternative kann die tanh-Funktion zum Einsatz kommen, welche zwar \(0\) abbilden kann, das Problem aber zu \(-1\) verlagert. Die ReLU-Funktion konnte nicht erfolgreich in das \ac{hnn} integriert werden, sodass dieses Problem vorerst ungelöst bleibt.

Ein weiteres Problem ergibt sich aus der kontinuierlichen Natur analoger Systeme. Während digitale Systeme mit diskreten Werten und stabilen Speicherzellen arbeiten, nutzen analoge Rechner kontinuierlichen Signale. Insbesondere die Speicherung der Gewichte stellt eine Herausforderung dar, da herkömmliche Integratoren mit der Zeit ihre Werte verlieren. Eine potenzielle Lösung wäre der Einsatz von Memristoren, welche sich bereits zur Umsetzung von Gewichten in neuronalen Netzen bewährt haben.

Zusätzlich gestaltet sich die Skalierung des \ac{c-ep} auf größere Netzwerke schwierig. Während das Verfahren für einfache \ac{hnn}e funktioniert, bleibt unklar, wie es sich auf tiefere Strukturen übertragen lässt. Analoge Systeme bringen sowohl Einschränkungen als auch Potenziale mit sich, die eine weiterführende Untersuchung erfordern.  Die Auswahl der Lernrate sowie des Einflußparameters spielt eine wichtige Rolle und wirkt sich auf die Stabilität und das Konvergenzverhalten des Algorithmus aus. Zu hohe Werte können zu Instabilitäten führen, zu niedrige Werte verlangsamen den Lernvorgang.
