\subsubsection{Bewertung der Simulationsumgebung und -werkzeuge}
\label{chap:Bewertung der Simulationsumgebung und -werkzeuge}

Die bereits von \citeauthor{Ulmann2022} empfohlene Software Simulink stellt alle nötigen Komponenten bereit, um analoge Schaltungen simulieren zu können \cite[vgl. S. 240]{Ulmann2022}. Darüber hinaus bietet Simulink auch eine Vielzahl weiterer Komponenten an, welche kein Äquivalent auf analoger Hardware besitzen, so \zb der "`Memory"'-Block. Es muss also beachtet werden, welche Komponenten genutzt werden können, um aus dem Modell später eine analoge Schaltung ableiten zu können. Grundsätzlich kann sich aber frei an den Bibliotheken "`Math Operations"' und "`Continuous"' bedient werden. Es muss auch die Abweichung der Blöcke aus Simulink von den analogen Rechenelementen beachtet werden. So fehlt \zb den Summierern aus Simulink der für Operationsverstärker typische Vorzeichenwechsel (vgl. Kapitel \ref{chap:Typische Komponenten und Bauweisen analoger Computer}) und die Blöcke aus Simulink geben grundsätzlich eine fehler- und störungsfreie Ausgabe, was nicht den realen Bedingungen entspricht.

Als Erweiterung der Software Matlab hat Simulink den Vorteil der Integration weiterer Komponenten, wie dem Matlab-Funktionsblock, siehe Kapitel \ref{chap:Auswahl und Grundlagen der Simulationssoftware}. Dieser Block hat in dieser Arbeit einige Anwendungsfälle gefunden, da somit \zb die Funktionsweise des \ac{hnn} validiert oder die Gewichtungen zu einer Matrix umgewandelt und somit zur Verwendung im \ac{hnn} vorbereitet werden konnten. Die Fähigkeit von Simulink, mit Vektoren und Matrizen zu arbeiten, hat sich hier auch bewährt, da somit nur ein einziges Neuron des \ac{hnn} modelliert werden muss, woraus ein Netzwerk mit beliebig vielen Neuronen erzeugt werden kann.

Ein für diese Arbeit schwerwiegendes Problem mit Simulink ist die Performance der Software in Verbindung mit dem Betriebssystem macOS. Die Konstruktion größerer Schaltungen (siehe \zb Anhang \ref{app:Umsetzung des Equilibrium-Propagation in Simulink}) führt zu einer trägen Benutzererfahrung und zieht somit den Prozess in die Länge. Bei der Ausführung komplexer Simulationen, wie hier ein Netzwerk mit mehr als drei Neuronen (siehe Kapitel \ref{chap:Validierung des Algorithmus durch Testläufe}), kommt das System auch an seine Grenzen, da diese bereits einige Sekunden bis wenige Minuten dauern. Das erschwert \zb eine effiziente Suche nach Hyperparametern zum Trainieren des Netzwerks, da diese bei steigender Komplexität des Netzwerks präzise gewählt werden müssen.
