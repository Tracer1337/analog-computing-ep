\subsubsection{Technische Herausforderungen und Lösungsansätze}
\label{chap:Technische Herausforderungen und Lösungsansätze}

Zur Umsetzung der Gewichte wurden bisher Integratoren genutzt, welche während der festen Phase mit einem konstanten Eingabewert \(0\) verschaltet werden und somit ihren Wert behalten. Nach \citeauthor{Ulmann2022} ist die Nutzung von Integratoren als temporäre Speicherzellen auch ein gängiges Vorgehen (\cite[vgl. S. 92]{Ulmann2022}), bei der Anwendung auf größere Netzwerke und damit längeren Inferenz-Phasen kann dieser Ansatz praktisch aber anhand der Natur von Kondensatoren, durch Leckströme ihre Ladung zu verlieren, zu Problemen führen. Wie in Kapitel \ref{chap:Auswahl der Referenzarbeit für das Netzwerkdesign} beschrieben, befassten sich bereits einige Arbeiten mit der Implementierung eines \gls{hopfieldnetzwerk} mit analogen Bauteilen. Eine häufig auftretende Bauweise beinhaltet den Memristor als zentrale Komponente, weshalb dieser auch hier zum Einsatz kommen könnte. So kann möglicherweise der akkumulierte Wert der Integratoren genutzt werden, um Memristoren zu programmieren, welche die Gewichte für die freie Phase speichern.

Eine Herausforderung bei der Konstruktion analoger Computer ist die notwendige Skalierung der einzelnen Werte, da diese typischerweise im Wertebereich \([-1;1]\) liegen müssen (siehe Kapitel \ref{chap:Typische Komponenten und Bauweisen analoger Computer}). Die Aktivierungsfunktion wurde unter diesen Voraussetzungen bereits passend gewählt, die Zustände des Netzwerks bzw. die Gewichte überschreiten dieses Limit aber. Um die Zustände auf \([-1;1]\) zu limitieren, muss eine Aktivierungsfunktion gewählt werden, welche im Bereich \(-1\leq x\leq 1\) in der Lage ist, alle Zielwerte abzubilden. Die im Kapitel \ref{chap:Strategien zur Fehlerbehebung und Optimierung} gezeigte ReLU Funktion könnte dieses Problem für Zielwerte im Bereich \([0;1]\) lösen. Die Gewichte könnten \zb um einen Faktor \(2\) skaliert und die Werte der Integratoren dadurch auf \([-1;1]\) beschränkt werden. In der Umsetzung muss dann nur noch sichergestellt werden, dass das Ergebnis der Multiplikation der Gewichte mit den Zuständen den Wertebereich \([-1;1]\) nicht überschreitet.

Das in dieser Arbeit implementierte \gls{c-ep} wurde auf nicht mehr als einen Zielwert pro Durchlauf angewandt. Soll das \gls{hopfieldnetzwerk} aber eine praktische Aufgabe lösen, muss im Lernprozess über eine Vielzahl an Zielwerten iteriert werden. Mithilfe des "`MNIST"' Datensatzes, bestehend aus 60.000 Bildern zu je 28x28 Pixeln, könnte das Netzwerk \zb auf die Erkennung handgeschriebener Ziffern trainiert werden \cite{Deng2012}. Hierfür benötigt es einen Prozess zur effizienten Eingabe der Zielwerte, welcher im einfachsten Fall durch einen hybriden Ansatz gelöst wird. Der Lernprozess pro Iteration könnte so aber weiterhin vollständig auf der analogen Recheneinheit durchgeführt werden.
