\subsubsection{Vergleich mit den Erwartungen und Zielsetzung}

Da für die kontinuierliche Gewichtsanpassung im \gls{c-ep} eine Differentialfunktion notwendig ist, welche für die von \citeauthor{Scellier2017} vorgestellte Energiefunktion nicht aufgestellt wurde, wurde als Alternative eine Annäherung an diesen Lernprozess mithilfe diskreter Zeitschritte versucht. Dies erfolgte durch eine zeitliche Verschiebung der Ausgabe des Netzwerks, um eine kontinuierliche Dynamik näherungsweise zu simulieren. Zur vollständigen Umsetzung des \gls{c-ep} ist jedoch eine mathematische Lösung erforderlich, die eine exakte Differentialgleichung für die Gewichtsanpassung liefert. Da in der aktuellen Literatur keine entsprechende Formulierung gefunden wurde, bleibt dieses Problem ungelöst.

Neben der fehlenden mathematischen Definition für die kontinuierliche Gewichtsanpassung zeigte sich eine weitere Herausforderung in der Wahl einer geeigneten Aktivierungsfunktion. Während in der Simulation eine Sigmoid-Funktion verwendet wurde, bleibt offen, ob diese die optimale Wahl für eine physikalische Implementierung darstellt. Eine geeignete Aktivierungsfunktion muss sowohl die Eigenschaften des Netzwerks als auch die physikalischen Grenzen analoger Rechner berücksichtigen und sollte eine effiziente Verarbeitung der Trainingsdaten ermöglichen.

Auch die Konvergenzgeschwindigkeit stellte eine relevante Fragestellung dar. Zwar konnten Fixpunkte des Netzwerks erreicht werden, jedoch variierte die Geschwindigkeit der Anpassung in Abhängigkeit von den gewählten Parametern. Zu hohe Lernraten führten zu Instabilitäten, während zu niedrige Lernraten den Lernprozess erheblich verlangsamten. Eine optimale Wahl der Lernrate sowie der Dauer der beiden Phasen ist entscheidend für die praktische Anwendbarkeit des \gls{c-ep} auf analoge Hardware.

Schließlich bleibt die Frage nach der Skalierbarkeit offen. Während die Implementierung in einem einfachen \gls{hopfieldnetzwerk} funktionierte, ist unklar, wie sich das \gls{c-ep} auf tiefere Netzwerke übertragen lässt. Analoge Hardware könnte dabei sowohl Einschränkungen als auch Vorteile bieten, die noch weiter untersucht werden müssen.
