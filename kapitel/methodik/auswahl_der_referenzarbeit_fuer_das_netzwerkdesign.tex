\subsection{Auswahl der Referenzarbeit für das Netzwerkdesign}
\label{chap:Auswahl der Referenzarbeit für das Netzwerkdesign}

Es haben sich bereits einige Arbeiten mit der Umsetzung eines \ac{hnn} auf Basis elektrotechnischer Schaltungen beschäftigt. So hat \zb \citeauthor{Guo2015} einen Analog-Digital-Wandler mithilfe von Memristoren zur Implementierung der Gewichte vorgestellt \cite{Guo2015}, welcher aber aufgrund der Komplexität der verwendeten Neuronen hier nicht zum Einsatz kommt. \citeauthor{Mathews2023} stellte \citeyear{Mathews2023} eine Implementierung eines \ac{hnn} auf einem \ac{fpaa} vor, womit der maximale Schnitt eines Graphen gefunden werden konnte. Diese Umsetzung eignet sich aber aufgrund des Einsatzes eines \ac{fpaa} auch nicht für diese Arbeit. Ein weiterer Einsatz von Memristoren als Gewichte konnte von \citeauthor{Hu2015} gezeigt werden, die in seiner Arbeit vorgestellten Neuronen arbeiten aber mit diskreten Werten (eins oder null) und das vorgestellte Netzwerk weicht vom ursprünglichen Design durch \citeauthor{Hopfield1982} ab \cite{Hu2015}. Als vierte und letzte Möglichkeit wurde wieder ein mithilfe von Memristoren implementiertes \ac{hnn} betrachtet, welches \citeyear{Hong2020} von \citeauthor{Hong2020} vorgestellt wurde und zur Bildwiederherstellung genutzt werden kann. Da diese Arbeit aber wieder, wie die von \cite{Guo2015}, komplexe Neuronen vorstellt und einiges an Vorwissen im Bereich Elektrotechnik zur Umsetzung und Verarbeitung in Simulink benötigt, ist sie auch ungeeignet.

\citeauthor{Scellier2017} haben in Ihrer Arbeit über \ac{ep} bereits die Energiefunktion eines leicht abgewandelten Hopfield-Netzwerks beschrieben und anhand dessen eine Gleichung zur Dynamik des Systems aufgestellt (siehe Kapitel \ref{chap:Theoretische Anwendung am Beispiel eines Hopfield-Netzwerks}). Um den Fokus dieser Arbeit auf die Implementierung des Lern-Algorithmus zu setzen und die Komplexität nicht zu übersteigen, wird ebendiese Dynamik in Simulink implementiert und anhand dessen das \ac{ep} validiert.
