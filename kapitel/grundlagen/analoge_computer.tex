\subsection{Analoge Computer}

\subsubsection{Geschichte zu analogen Computern}

Die frühen mechanischen analogen Computer stellen eine faszinierende Entwicklung in der Geschichte der Rechentechnik dar. Ein bemerkenswertes Beispiel ist das Astrolabium, das etwa um 150 v. Chr. entstand. Diese einfachen Inklinometersysteme dienten zur Modellierung grundlegender Aspekte der sphärischen Astronomie\footcite[Vgl. ][9]{ulmann2022analog}. Das Astrolabium fand vor allem in der Navigation Verwendung, wo es half, die Position von Himmelskörpern zu bestimmen, wenngleich seine Präzision begrenzt war\footcite[Vgl. ][9]{ulmann2022analog}.

Ein weiteres bedeutendes antikes Gerät ist der Antikythera-Mechanismus, der 1900 in einem römischen Schiffswrack entdeckt wurde und aus etwa 100 v. Chr. stammt. Er gilt als eines der komplexesten mechanischen und mathematischen Geräte der Antike und konnte die Bewegungen von Himmelskörpern modellieren sowie Sonnen- und Mondfinsternisse vorhersagen\footcite[Vgl. ][9-10]{ulmann2022analog}.

Im Verlauf der technischen Entwicklung wurden Gleitkommarechner konstruiert, mechanische Geräte, die zur Lösung komplexer mathematischer Probleme, wie der Berechnung von Bombenflugbahnen und Feuerleitsystemen, eingesetzt wurden. Diese fanden auch in friedlichen Anwendungen wie der Gezeitenberechnung Verwendung\footcite[Vgl. ][9]{ulmann2022analog}.

Mit dem Fortschritt der Technik entstanden die ersten elektronischen analogen Computer, die von Helmut Hoelzer in Deutschland und George A. Philbrick in den USA entwickelt wurden. Diese Computer wurden primär für militärische Anwendungen wie Flugbahn- und Feuerleitsysteme genutzt\footcite[Vgl. ][41]{ulmann2022analog}. Ein Beispiel für einen frühen elektronischen Analogrechner ist Hoelzer’s Mischgerät, das während des Zweiten Weltkriegs in Deutschland entwickelt wurde und Elektronenröhren zur Durchführung von Berechnungen verwendete\footcite[Vgl. ][42]{ulmann2022analog}.

George A. Philbrick spielte eine bedeutende Rolle in der Weiterentwicklung und Verbreitung elektronischer Analogrechentechnologie. Er führte kommerzielle Operationsverstärker ein und entwickelte in den 1950er Jahren modulare elektronische Analogcomputer, was einen großen Einfluss auf die Standardisierung und Verbreitung dieser Technologie hatte\footcite[Vgl. ][136]{ulmann2022analog}.

Ein weiteres monumentales Beispiel ist der Caltech-Computer, ein elektronischer Analogrechner, der zwischen 1946 und 1947 entwickelt wurde und etwa 15 Tonnen wog. Dieser Computer wurde zur Lösung komplexer mathematischer und wissenschaftlicher Probleme, einschließlich Differentialgleichungen, entwickelt\footcite[Vgl. ][69]{ulmann2022analog}.

Diese Entwicklungen zeigen, wie mechanische und elektronische analoge Computer bedeutende Fortschritte in der Rechentechnik ermöglichten, sowohl in militärischen als auch in zivilen Anwendungen.

\subsubsection{Typische Komponenten und Bauweisen analoger Computer}

\subsubsection{Aufbau von Schaltkreisen zur Lösung von Differenzialgleichungen}
