\subsubsection{Geschichte zu analogen Computern}

Die Geschichte des analogen Computers reicht bis in die Antike zurück. So wurde im Jahr 1900 der Antikythera-Mechanismus in einem Schiffswrack entdeckt, der aus etwa 100 v. Chr. stammt. Er gilt als eines der komplexesten mechanischen und mathematischen Geräte der Antike und konnte die Bewegungen von Himmelskörpern modellieren sowie Sonnen- und Mondfinsternisse vorhersagen. (\cite[vgl. S. 9 f.]{Ulmann2022})

Im Verlauf der technischen Entwicklung wurden Gleitkomma-Rechner konstruiert, mechanische Geräte, die zur Lösung komplexer mathematischer Probleme, wie der Berechnung von Bombenflugbahnen und Feuerleitsystemen, eingesetzt wurden. Diese fanden auch in friedlichen Anwendungen wie der Gezeiten-Berechnung Verwendung. (\cite[vgl. S. 9]{Ulmann2022})

Mit dem Fortschritt der Technik entstanden die ersten elektronischen analogen Computer, die von Helmut Hoelzer in Deutschland und George A. Philbrick in den USA entwickelt wurden. Diese Computer wurden primär für militärische Anwendungen wie Flugbahn- und Feuerleitsysteme genutzt. Ein Beispiel für einen frühen elektronische Analogrechner ist Hoelzers Mischgerät, das während des zweiten Weltkriegs in Deutschland entwickelt wurde und Elektrohnenröhren zur Durchführung von Berechnungen verwendete. (\cite[vgl. S. 41 f.]{Ulmann2022})

Ein weiteres bedeutendes Gerät war der Caltech-Computer, ein elektronischer Analogrechner, der zwischen 1946 und 1947 entwickelt wurde und etwa 15 Tonnen wog. Dieser Computer wurde zur Lösung komplexer mathematischer und wissenschaftlicher Probleme, einschließlich Differentialgleichungen, entwickelt. (\cite[vgl. S. 69]{Ulmann2022})

George A. Philbrick spielte eine bedeutende Rolle in der Weiterentwicklung und Verbreitung elektronischer Analogrechner. Er führte kommerzielle Operationsverstärker ein und entwickelte in den 1950er Jahren modulare elektronische Analogrechner, was einen großen Einfluss auf die Standardisierung und Verbreitung dieser Technologie hatte. (\cite[vgl. S. 136]{Ulmann2022})
