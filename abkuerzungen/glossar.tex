\newglossaryentry{glossar}{name={Glossar},description={In einem Glossar werden Fachbegriffe und Fremdwörter mit ihren Erklärungen gesammelt.}}
\newglossaryentry{glossaries}{name={Glossaries},description={Glossaries ist ein Paket was einen im Rahmen von LaTeX bei der Erstellung eines Glossar unterstützt.}}
\newglossaryentry{dense_layer}{name={Dense Layer},description={Ebene eines neuronalen Netzes, in der jedes Neuron mit jedem Neuron der vorherigen Ebene verbunden ist}}
\newglossaryentry{fully_connected_layer}{name={Fully Connected Layer},description={Siehe "`Dense Layer"'}}
\newglossaryentry{bias_neuron}{name={Bias-Neuron},description={Zusätzliches Neuron, welches Konstant den Wert 1 ausgibt. Durch gewichtete Verbindungen kann jedem Neuron der nachfolgenden Ebene ein Bias-Wert zugewiesen werden.}}
\newglossaryentry{perceptron}{name={Perceptron},description={Sammlung an TLUs auf einer einzelnen Ebene}}
\newglossaryentry{tlu}{name={TLU},description={Threshold Logic Unit; Berechnet die gewichtete Summe seiner Eingabewerte und gibt abhängig von der Überschreitung eines Schwellenwerts entweder 0 oder 1 aus.}}
\newglossaryentry{mlp}{name={MLP},description={Multilayer-Perceptron; Eine Zusammensetzung an Perceptrons mit einer Eingabe-Ebene, mindestens einer versteckten Ebene und einer Ausgabe-Ebene.}}
\newglossaryentry{fnn}{name={FNN},description={Feedforward Neural Network; Neuronales Netz, in dem Signal nur in eine Richtung geleitet werden.}}
\newglossaryentry{dnn}{name={DNN},description={Deep Neuronal Network; Neuronales Netz mit vielen versteckten Ebenen.}}
\newglossaryentry{cnn}{name={CNN},description={Convolutional Neural Network; Neuronales Netz mit Convolutional-Ebenen, welche nur mit jeweils einem Ausschnitt der vorherigen Ebene verbunden sind.}}
\newglossaryentry{rnn}{name={RNN},description={Recurrent Neural Network; Neuronales Netz mit rückwärtigen Verbindungen, Hauptbestandteil ist hier die Speicherzelle.}}